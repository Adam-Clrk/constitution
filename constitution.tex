\documentclass[12pt]{constitution}
\usepackage{color,soul}
\usepackage[normalem]{ulem}

\begin{document}

% TITLE
% =====
\begin{titlepage}
\begin{center}

\vspace*{5cm}
\includegraphics[width=0.4\textwidth]{logo-transparent.png}\\
\vspace{0.5cm}
{\fontsize{40}{48}\robotoLight{University of Southampton}}\\
\vspace{0.15cm}
{\fontsize{40}{48}\robotoLight{\textbf{LGBTQ+} Society}}\\
\vspace{0.4cm}
{\fontsize{30}{36}\selectfont Constitution}\\
\end{center}
\end{titlepage}

% CONTENT
% =======

\article{Name}
\label{article:name}

The official name of the Society herein is the LGBTQ+ Society, which is referred to as the University of Southampton LGBTQ+ Society for national distinction.

% ---------------------------------------------------------

\article{Objects}
\label{article:objects}

The objectives of the Society, the `Objects', are to support and further the interests of members of the University of Southampton (herein the University) who are part of the LGBTQ+ community, by the following means:

\begin{enumerate}
    \item Support and represent LGBTQ+ members of the University in coordination with the the Students' Union Equality \& Diversity Committee.
    \item Ensuring a safe environment for LGBTQ+ students to work and socialise at and around the University campuses, as well as within our online groups.
    \item Actively pursuing charitable and political campaigns to further the well-being of LGBTQ+ people.
    \item Representing the views of the Society’s members, by working with local and national bodies and participating in relevant events, where it is practical to do so.
    \item Improving awareness of the health and political issues that affect the Society’s members, and the services available to support them.
    \item Raising awareness of LGBTQ+ people in the wider community, and the issues that affect them.
\end{enumerate}

% ---------------------------------------------------------

\article{Affiliation}
\label{article:affiliation}

The Society shall be affiliated to the University of Southampton Students’ Union, hereinafter referred to as the Students’ Union.

% ---------------------------------------------------------

\article{Membership}
\label{article:membership}

Membership to the Society is subject to the following rules:

\begin{enumerate}
    \item Membership is open to natural persons, and is not transferable to anyone else.
    \item Membership is constituted in the following categories:
    \begin{enumerate}
        \item Full, open to any Full Members of the Students' Union.
        \item Associate, open to Associate and Temporary Members of the Students’ Union, and to those students of the University who have exercised their right not to be members of the Students’ Union.
    \end{enumerate}
    \item The Society will not charge a fee for membership.
    \item Only Full Members are entitled to become an elected official, or to propose, discuss and vote at a General Meeting. These are the sole privileges afforded to the Full Members over any other category of Membership.
    \item A single opt-in membership list is maintained, in accordance with Article \ref{article:equality-confidentiality}, `Equality and Confidentiality'.
\end{enumerate}

% ---------------------------------------------------------

\article{Equality and Confidentiality}
\label{article:equality-confidentiality}

\begin{enumerate}
    \item The attendance of a person, to any Society meeting or event, implies nothing about their sexual orientation nor gender identity. Whenever there is doubt that all attendees are aware of this policy, they should be reminded.
    \item The Committee shall maintain a single opt-in membership list, which may be used for determining voter or candidate eligibility, quoracy at General Meetings, safeguarding of members and where otherwise required by law. All other uses of personally identifiable member information are prohibited without the express consent of the members involved. The list shall be purged annually, and clear instructions shall be given to Society Members on how to relist. The Committee shall hold these records in line with data protection legislation, in particular ensuring appropriate technical and organisational measures are put in place to prevent misuse or unauthorised access and shall publish a Member Data Policy explaining clearly to members how their data is used and secured.
    \item People of all ages are welcome to join, but the Secretary must record the number of Members under the age of 18, and this number should be considered when organising events and meetings involving age restrictions.
    \item After every Annual General Meeting, all minutes from the previous committee shall be anonymised such that full names are not used.
\end{enumerate}

% ---------------------------------------------------------

\article{Elected Officials}
\label{article:officers-committee}

\begin{enumerate}
    \item The administration and management of the Society is the responsibility of its Committee. The Committee consists of the Officers of the Society and Ordinary Members.
    \item \label{subclause:elected-officials} The Society shall have the following elected officials:
    \begin{enumerate}
        \item President: The President is the figurehead of the Society to all external interests, and is responsible for the Society’s policies and actions throughout their term. It is essential that the President delegates tasks to the other committee members wherever possible. They should also ensure the officers’ accountability to Members, the Committee and the Students’ Union.
        \begin{enumerate}
            \item It is desirable that the President have at least one year experience on a Students’ Union society or club committee, but this is not essential.
        \end{enumerate}
        \item Secretary: The Secretary oversees the administration of the Society, minutes all General Meetings and Meetings of the Committee, and ensures that Society and Students’ Union policy is followed in all proceedings. They are also responsible for the Society’s communications with external parties.
        \item Treasurer: The Treasurer oversees the financing of the Society, submits funding applications to the Students’ Union and maintains the accounts for the Society. They must be involved whenever money is handled by the Society. They should also be involved when organising any sponsorship arrangements with external parties.
        \item Vice-President: The Vice-President supports the President in the organisation and management of the Committee and Society, with a particular responsibility for engagement and diversity. This role should encompass the spirit of all former engagement and diversity roles within the committee and ensure that every area of the society remains committed to serving minority groups within the LGBTQ+ community.
        \item Welfare Officers: There shall be two Welfare Officers. The Welfare Officers are essential in leading the Society’s support service. They are responsible for the welfare material delivered by the Society, including information leaflets and awareness talks. They should make Members aware of the support services offered by the University, the Students’ Union and local LGBTQ+ support services.
        \item Activities Officer: The Activities Officer is responsible for coordinating the organisation of events, and developing ideas to ensure a balance of activities that interest all Society Members. They are expected to lead the organisation of events with the support of other committee and Society Members. Special consideration should be given to ensuring events are inclusive of all Society Members when organising events.
        \item Technical Officer: The Technical Officer maintains the essential computing services of the Society, including the website and committee email services.
        \begin{enumerate}
            \item It is desirable that the Technical Officer have experience building, running and maintaining a website, but this is not essential.
        \end{enumerate}
        \item Publicity Officer: The Publicity Officer coordinates promotion of the Society’s activities and services to Society Members and the wider community. This includes online publicity through social media and the website, working closely with the Activities Officer.
        \item Ordinary Members: There shall be two Ordinary Members who are unaffiliated with any groups in clause \ref{subclause:groups}. Additional Ordinary Members may be needed, subject to clauses \ref{subclause:groups} and \ref{subclause:groups-elect}. The Ordinary Members shall support the Officers in their endeavours and take on responsibilities as needed.
    \end{enumerate}

    \item It is essential that the Society has a President, a Treasurer and a Secretary in order to exist and function. Without these, the terms of the Students’ Union's affiliation are null and void.
    
    \item \label{subclause:groups} The Committee should aim to have at least one individual who self-identifies in one of the following groups: 
    \begin{enumerate}
        \item Lesbian
        \item Gay
        \item Bisexual / Pansexual
        \item Trans / Non-Binary
        \item Asexual
        \item Aromantic
        \item BAME
        \begin{enumerate}
            \item Using the Office for National Statistics' definition of `minority ethnicity', an individual who is `BAME' is one who self-identifies in an ethnic group other than White British.
        \end{enumerate}
        \item Students at the Winchester School of Art
    \end{enumerate}
    
    \item \label{subclause:groups-elect} If a Committee that is newly formed after an Annual General Meeting (AGM) does not have any individuals who identify in one of the groups listed in sub-clause (\ref{subclause:groups}), an Extraordinary General Meeting (EGM) shall be called to elect an Ordinary Member who shall represent the interests of the unrepresented group. 
    
    \item An elected official shall cease to hold office if they:
    \begin{enumerate}
        \item cease to be a Full Member of the Society.
        \item resign by notice to the Society, or
        \item are removed from office by a resolution of the Members in General Meeting or a Meeting of the Committee, in accordance with Article \ref{article:disciplinary-action}, `Disciplinary Action'.
    \end{enumerate}
    
    \item If any Committee position is vacant, the Committee must delegate these responsibilities amongst themselves.
    
    \item An Extraordinary General Meeting (EGM) shall be called as soon as is reasonably possible in order to fill a vacant elected position.
    \begin{enumerate}
        \item If the EGM fails to fill a vacant elected position, the committee will only need to run another EGM when interest in the role becomes apparent.
    \end{enumerate}
\end{enumerate}

% ---------------------------------------------------------

\article{General Meetings}
\label{article:general-meetings}

\begin{enumerate}
    \item General Meetings constitute the Group's highest decision-making body, subject to the provisions of this Constitution.
    \item The Society must hold an Annual General Meeting (AGM) in each academic year and not more than fifteen months may elapse between successive AGMs.
    \item A General Meeting that is not an Annual General Meeting is called an Extraordinary General Meeting (EGM).
    \item An EGM can be called in exceptional circumstances by:
    \begin{enumerate}
        \item the President
        \item the Secretary
        \item a petition to the President naming at least six Society members, and giving the reason for the meeting.
    \end{enumerate}
\end{enumerate}

% ---------------------------------------------------------

\article{Proceedings of General Meetings}
\label{article:proceedings-general-meetings}

\begin{enumerate}
    \item Notice:
    \begin{enumerate}
        \item The minimum period of notice required to hold an Annual General Meeting is fourteen days. The minimum period of notice required to hold an Extraordinary General Meeting is seven days.
        \item The notice must specify the date, time and place of the General Meeting, and an agenda for the General Meeting.
        \item If the General Meeting is to be an AGM, the notice must say so, and must invite nominations in accordance with Article~\ref{article:appointment-committee}, `Appointment of the Committee'.
        \item Notice must be given to all members of the Society.
    \end{enumerate}

    \item Chairing:
    \begin{enumerate}
        \item General Meetings shall usually be chaired by the person who has been elected as President.
        \item If there is no such person, they are not present within fifteen minutes of the time appointed for the General Meeting, or the President steps down as chair of the meeting, another member of the Committee shall take the role of Chair.
    \end{enumerate}

    \item Associate Members of the Society may speak at a General Meeting only with the permission of the meeting.

    \item Conduct:
    \begin{enumerate}
        \item Every Full Member present at a General Meeting, with the exception of the Chair, shall be entitled to one vote upon every voting matter. In the case of an equality of votes, the Chair shall have a casting vote.
        \item All votes are subject to a quorum of 20\% of the Full Members (i.e. 20\% of the Full Members included on the confidential Membership List).
        \item Decisions may only be made by at least a simple majority of votes at a quorate General Meeting.
        \item All voting on resolutions shall be by secret ballot, at the discretion of the Chair.
        \item There shall be no absentee voting.
    \end{enumerate}

    \item Minutes:
    \begin{enumerate}
        \item Minutes must be taken of all proceedings at a General Meeting, including the decisions made and where appropriate the reasons for the decisions.
        \item Minutes of a General Meeting shall be made available to all Members within seven days.
        \item Minutes shall be taken by the Secretary. When the Secretary is not present, minutes shall be taken by another member of the Committee delegated by the Chair of the meeting.
    \end{enumerate}

    \item Reports:
    \begin{enumerate}
        \item If the General Meeting is an AGM, the Chair may invite any of the Committee to offer a report of their activities whilst in office.
        \item The Treasurer must present the Society's accounts to the Members at the AGM.
        \item If the Treasurer is not present, another committee member should present the Society's accounts on behalf of the Treasurer.
    \end{enumerate}

    \item Resolutions:
    \begin{enumerate}
        \item Any Full Member may propose a resolution to be discussed and voted upon at a General Meeting.
    \end{enumerate}
    
    \item Alternative Arrangements:
    \begin{enumerate}
        \item In times of national emergency, university closure, or other such crisis, the Committee may, by majority vote at a meeting of the Committee, decide to make alternative arrangements for voting without the requirement for an in-person Annual General Meeting (AGM). The suggested format for such events is as follows but must be agreed by the Committee in advance.
        \begin{enumerate}
            \item A two-week nomination period, in which society members who wish to stand for a committee position may submit their manifesto so the current Society President, who will keep them confidential.
            \item A 48-hour voting window where votes may be made remotely using the safest and most secure method available, as decided by the Committee in advance.
            \item A remote results announcement event with the usual accounts and summaries that would ordinarily be presented at an in-person AGM.
        \end{enumerate}
    \end{enumerate}
\end{enumerate}

% ---------------------------------------------------------

\article{Meetings of the Committee}
\label{article:meetings-committee}
\begin{enumerate}
    \item The Committee may regulate their proceedings as they think fit, subject to the provisions of this Article.
    \item Any member of the Committee may request the Secretary to call a Meeting of the Committee.
    \item The Secretary must call a Meeting of the Committee if requested to do so by a member of the Committee.
    \item Meetings of the Committee shall usually be chaired by the person who has been elected as President.
    \item The quorum for a Meeting of the Committee shall be three members of the Committee or 50\% of the Committee, whichever is higher.
    \item No decision may be made by a Meeting of the Committee unless a quorum is present at the time the decision is made.
    \item Every member of the Committee shall be entitled to one deliberative vote upon every voting matter. In the case of an equality of votes, the vote of the Chair shall be considered the casting vote.
    \item Decisions may only be made by at least a simple majority of votes at a quorate Meeting of the Committee.
    \item There shall be no absentee voting.
    \item The Committee may, at its discretion, invite individuals who are not elected officials to the meeting.
    \item Minutes must be taken of all proceedings at a Meeting of the Committee, including the decisions made.
\end{enumerate}

% ---------------------------------------------------------

\article{Appointment of the Committee}
\label{article:appointment-committee}

\begin{enumerate}
    \item The Full Members of the Society in General Meeting shall appoint the Officers of the Committee and Ordinary Members by election.
    \begin{enumerate}
        \item Elections for the Committee shall be held at an Annual General Meeting. By-elections for vacant offices shall be held at an Extraordinary General Meeting.
        \item Scotland's rules for Single Transferable Vote, as defined in the Scottish Local Government Elections Order 2007, shall be the electoral system for all elections.
        \item In all elections Re-Open Nominations, `RON', shall be a candidate.
        \begin{enumerate}
            \item An election for a role with one seat (e.g. President, Secretary) yielding a result of `RON' shall be re-run as a by-election.
            \item If an election for a role with more than one seat (e.g. Welfare Officers) elects `RON', the counting of the votes for the role will then cease. The seat won by `RON' and any remaining seats will be re-run as a by-election.
        \end{enumerate}
        \item Each role shall be elected in the order that the roles are specified in sub-clause~(\ref{subclause:elected-officials}) of Article~\ref{article:officers-committee}, `Elected Officials'.
        \item If an individual stands for multiple roles and is elected to a role, they will be unable to be elected to the roles that are subsequently elected.
        \begin{enumerate}
            \item All individuals who are ineligible to be elected to a role as a result of this sub-clause shall have their votes redistributed to their next preference.
        \end{enumerate}
        \item In accordance with Article~\ref{article:equality-confidentiality}, `Equality and Confidentiality', no candidate may campaign in public spaces.
        \item Candidates' campaigning activities shall be limited solely to those organised for the purposes of the election campaign by the Committee.
        \item The Committee must not favour any candidate over any other candidate in its organisation of campaign events. All candidates must be invited to all campaign events, and the Committee must organise such events with a good-faith view to ensuring that all candidates are able to attend.
        \item The Committee shall, before nominations open, appoint a Returning Officer to oversee the election. The Returning Officer will be responsible for the interpretation and enforcement of the Constitution in relation to the election and must not be a candidate standing for election in any capacity.
    \end{enumerate}
    
    \item Individuals who wish to stand for an Ordinary Member role representing a group listed in Article~\ref{article:officers-committee}, `Elected Officials', must self-identify or have previously self-identified as the group that they wish to represent.

    \item The count for elections shall be conducted publicly by the Chair of the General Meeting, who must do so accurately. Should the Members in General Meeting be dissatisfied with the accuracy of the count, they may resolve as a Point of Order to have the election re-counted or, if they remain dissatisfied, re-run as a by-election.
    
    \item The count for elections may be conducted electronically. If the count is conducted electronically, the software used to conduct the count must be made publicly available at the time that the General Meeting is called.

    \item \begin{enumerate}
        \item A member of the Committee shall assume office with effect from the conclusion of the General Meeting of their appointment.
        \item A member of the Committee shall retire with effect from the conclusion of the AGM next after their appointment, but shall be eligible for re-election at that AGM.
    \end{enumerate}

    \item The Committee must update their committee information on the Student Societies Hub provided by the Students' Union at \url{www.susu.org} (or failing that inform the Students' Union's Student Groups Officer) within seven days.
    
    \item A retiring member of the Committee must transfer all relevant information and documentation to their newly-elected counterpart, or to the President, within fourteen days.
\end{enumerate}

% ---------------------------------------------------------
\article{Financial Management}
\label{article:financial-management}

\begin{enumerate}
    \item The elected officials of the Society are jointly liable for the proper management of the Society's finances and ensuring that the Students' Union’s financial regulations are followed.
    \item The income and property of the Group must be applied solely towards the promotion of the objects.
    \item The members of the Committee are entitled to be reimbursed from the property of the Group or may pay out of such property only for reasonable expenses properly incurred by them when acting on behalf of the Group.
    \item The accounts of the Group, as maintained by the Treasurer, must be made available to the Students' Union upon request.
    \item The financial concerns of the Society are entirely its own and do not affect, nor are they affected by, the Students' Union’s financial concerns, except for the Society’s allocated budget. In particular, if the Society finds itself in debt, regardless of the members responsible, it is the Society’s own concern to repay the debt.
    \item Equally, the Students' Union should only demand money from the Society as payment for a specific service that the Society has requested. In particular, no affiliation fee exists at the time of this Constitution’s passing; this document is void should this situation change.
\end{enumerate}

% ---------------------------------------------------------

\article{Irregularities and Saving Provisions}
\label{article:irregularities-saving-provisions}

\begin{enumerate}
    \item Subject to sub-clause (b) of this Article, all acts done by a Meeting of the Committee shall be valid notwithstanding the participation in any vote of a member of the Committee:
    \begin{enumerate}
        \item who was disqualified from holding office;
        \item who had previously retired or who had been obliged by this Constitution to vacate office;
        \item who was not entitled to vote on the matter, whether by reason of a conflict of interests or otherwise.
    \end{enumerate}

    \item Sub-clause (a) of this Article does not permit a member of the Committee to keep any benefit that may be conferred upon them by a resolution of the Committee if the resolution would otherwise have been void, or if the Committee has not complied with Article~\ref{article:conflicts-interest-loyalities}, `Conflicts of Interests and Conflicts of Loyalties'.
    \item The Members in General Meeting may only invalidate, as a Point of Order, a resolution or act of:
    \begin{enumerate}
        \item the Committee;
        \item the Members in a General Meeting;
    \end{enumerate}
    if it may be demonstrated that a procedural defect in the same has materially prejudiced a Member of the Society.
\end{enumerate}

% ---------------------------------------------------------

\article{Conflicts of Interest and Conflicts of Loyalties}
\label{article:conflicts-interest-loyalities}

\begin{enumerate}
    \item A Member of the Committee must:
    \begin{enumerate}
        \item declare the nature and extent of any interest, direct or indirect, which they have in any decisions of a Meeting of the Committee or in any transaction or arrangement entered into by the Society which has not been previously declared;
        \item absent themself from any discussions of the Committee in which it is possible that a conflict will arise between their duty to act solely in the interests of the Society and any personal interest, including but not limited to any personal financial interest.
    \end{enumerate}

    \item Any member of the Committee absenting themself from any discussions in accordance with this Article must not vote or be counted as part of the quorum in any decision of the Committee on the matter.
\end{enumerate}

% ---------------------------------------------------------

\article{Disciplinary Action}
\label{article:disciplinary-action}

\begin{enumerate}
    \item Disciplinary action may be taken against any Member of the Society as a consequence of conduct:
    \begin{enumerate}
        \item detrimental to the reputation of the Society or the Students' Union.
        \item opposed to the objects of the Society (see Article \ref{article:objects}) or the Students' Union.
        \item in contravention of any provision of this Constitution.
    \end{enumerate}
    
    \item Disciplinary action that may be taken against any Member may be, but is not limited to:
    \begin{enumerate}
        \item issue of a formal written warning.
        \item partial or total ban from certain Society activities.
        \item disqualification from becoming an elected official.
        \item removal of an elected official from office.
        \item temporary or permanent revocation of Membership.
        \item referral of the complaint to the Students' Union's Disciplinary Committee.
    \end{enumerate}

    \item It is the right of the subject of the complaint to choose to have the disciplinary matter heard by either the Members in General Meeting, or a Meeting of the Committee.  Either shall have the power to take disciplinary action, including but not limited to those measures set out in paragraphs (i) - (vi) inclusive in sub-clause (b) of this Article.

    \item Any disciplinary hearing must be conducted in an impartial, balanced, and fair manner, considering all representations on the matter.

    \item All disciplinary action must be subject to prior discussion with the Students' Union's Student Groups Officer.

    \item Members subject to disciplinary action have the right of appeal to the Students' Union's Student Groups Committee.

    \item A full report of all disciplinary action taken by the Society in the previous year must be presented at the Annual General Meeting.
\end{enumerate}

% ---------------------------------------------------------

\article{Affiliation to External Organisations}
\label{article:affiliation-external-organisations}

\begin{enumerate}
    \item The Society may only become an affiliate of an external organisation if:
    \begin{enumerate}
        \item the aims of that organisation are in line with those of the Society;
        \item the Members derive a direct benefit from the affiliation;
        \item no Policy of the Students' Union is breached by the affiliation;
        \item a resolution to affiliate is passed by the elected officials in a Meeting of the Committee.
    \end{enumerate}

    \item The Society's affiliation to an external organisation shall immediately lapse if a resolution to disaffiliate is passed by the Members in General Meeting or a Meeting of the Committee.

    \item All external affiliations and disaffiliations must be reported to the Students' Union's Student Groups Committee within seven days.
    
    \item For the avoidance of doubt, the Students' Union is not an external organisation for the purposes of this Article.
\end{enumerate}

% ---------------------------------------------------------

\article{Amendment to the Constitution}
\label{article:amendment-constitution}

\begin{enumerate}
    \item The Society may amend any provision contained in this Constitution provided that:
    \begin{enumerate}
        \item amendments do not:
        \begin{enumerate}
            \item alter the objects in such a way that undermines or works against the previous objects of the Group;
            \item retrospectively invalidate any prior act of the Members in General Meeting or a Meeting of the Committee;
        \end{enumerate}

        \item a resolution to amend a provision of this Constitution is passed by at least a two-thirds majority of the Full Members present at a General Meeting;
        \item a copy of the resolution amending this Constitution is sent to the Students' Union within seven days of it being passed;
        \item the resolution is ratified by the Students' Union's Student Groups Committee.
    \end{enumerate}

    \item The interpretation of this Constitution shall be with the Committee, except that during a General Meeting or a Meeting of the Committee the Chair shall have this responsibility.  The Members in General Meeting may resolve to revise any interpretation made by the Committee or a Chair as a Point of Order.
    \item The provisions of this Constitution shall be subordinate to those of the Articles, Rules, By-Laws and Policies of the Students' Union.
    \item The Committee and the Students' Union shall retain a copy of this Constitution, which the Committee must make available to Members upon request.
\end{enumerate}

% ---------------------------------------------------------

\article{Dissolution}
\label{article:dissolution}

\begin{enumerate}
    \item If the Members resolve to dissolve the Group, the Committee will remain in office and be responsible for winding up the affairs of the Group in accordance with this Article.
    \item A resolution to dissolve the Group must be passed by at least a two-thirds majority of the Full Members present at a General Meeting;
    \item The Committee must collect in all the assets of the Group and must pay or make provision for all the liabilities of the Group.
    \item The Committee must apply any remaining property or money:
    \begin{enumerate}
        \item directly for the objects;
        \item by transfer to any Group or Societies for purposes the same as or similar to the Group;
        \item in such other manner as the Students' Union's Student Groups Committee may approve in writing in advance.
    \end{enumerate}

    \item The Members may pass a resolution before or at the same time as the resolution to dissolve the Group specifying the manner in which the Committee are to apply the remaining property or assets of the Group.  The Committee must comply with such a resolution if it is consistent with the provisions of this Article.
    \item In no circumstances shall the net assets of the Group be paid to or distributed among the Members of the Group.
    \item The Committee must ensure the register and all other data held by the Group are securely destroyed upon the dissolution of the Group.
    \item The Committee must notify the Students' Union within seven days that the Group has been dissolved.  If the Committee are obliged to send the Group's accounts to the Students' Union for the accounting period which ended before its dissolution, they must send the Students' Union the Group's final accounts.
\end{enumerate}

% ---------------------------------------------------------

\article{Interpretation}
\label{article:interpretation}

In this Constitution:
\begin{enumerate}
    \item `The University' means `the University of Southampton'.
    \begin{enumerate}
        \item `University term' and `academic year' have the definitions set out in the University Calendar.
    \end{enumerate}

    \item `Financial benefit' means a benefit, direct or indirect, which is either money or has monetary value.
    \item `The Students' Union' mean `The University of Southampton Students' Union'.
    \begin{enumerate}
        \item `Articles', or `Articles of the Students' Union' mean the Students' Union's Articles of Association. `Rules' and `Policies' have the definitions set out in the Articles. `By-Laws' has the definition set out in the Rules.
    \end{enumerate}
    
    \item `Elected official' means an individual who is elected as either an Officer of the Committee or as an Ordinary Member.
\end{enumerate}

% ---------------------------------------------------------

\article{Declaration}
\label{article:declaration}

The Society Ratified Changes to This Constitution.

Date: 2021-04-26

President: Kestral Gaian

Secretary: Oisin Clarke-Willis

\end{document}

